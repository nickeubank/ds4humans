

\documentclass[12pt]{article}


\usepackage[T1]{fontenc}
\usepackage{amsfonts, amsmath, amssymb}
\usepackage{authblk}
\usepackage{multirow}
\usepackage{epsfig}
\usepackage{subfigure}
\usepackage{subfloat}
\usepackage{graphicx}
\usepackage{lscape}
\usepackage{amsmath}

\usepackage{amssymb}
\usepackage{gensymb}
\usepackage{tabularx}

\usepackage{booktabs}
\usepackage{longtable}
\usepackage{verbatim, rotating, paralist}
\usepackage{enumerate}
\usepackage{natbib}
\usepackage{multibib}
\usepackage{pdfsync}
\usepackage{latexsym}
\usepackage{amsthm}


\usepackage{stmaryrd}
\usepackage{dsfont}
\usepackage{hyperref}
\usepackage{bbm}
\usepackage{mathtools}

\usepackage{parskip}
\usepackage{anysize, indentfirst, setspace}
\usepackage[right=2.1cm, left=2.1cm, top=2.25cm, bottom=2.25cm]{geometry}
\usepackage{tikz}
\usepackage{epigraph}
\usepackage{csquotes}
\usepackage{appendix}
\usepackage{enumitem}
\usepackage{rotating}
\usepackage{changepage}


\renewcommand{\topfraction}{.85}
\renewcommand{\bottomfraction}{.7}
\renewcommand{\textfraction}{.15}
\renewcommand{\floatpagefraction}{.66}
\renewcommand{\dbltopfraction}{.66}
\renewcommand{\dblfloatpagefraction}{.66}

\usetikzlibrary{arrows,shapes,backgrounds,positioning,patterns,decorations.pathreplacing}
\newenvironment{shift}{\begin{adjustwidth}{1cm}{}}{\end{adjustwidth}}
\setlist{nosep}
\newcites{sec}{Appendix References}

%\renewcommand{\footnotelayout}{\doublespacing}
%\linespread{1.5}

\title{Solving Real Problems with Data \\ An Intermediate Data Science Text}

%-----------------------------------BEGIN DOCUMENT--------------------------------%
\begin{document}

\setlength{\parindent}{0.0in}
\setlength{\parskip}{.125in}


\date{\today }
%\date{April 25, 2021 }
%\date{November 24, 2020 }


\maketitle
\section*{1. The 30-Second Sell}\label{the-30-second-sell}

In my six years working with students in the Duke Masters of
Interdisciplinary Data Science (MIDS) program --- of which I am now the
Faculty Director --- I have found that the hardest part of training data
scientists preparing them for the transition from doing technically
challenging but well-structured classroom exercises to applying data
science tools effectively when faced with the ambiguities of real world
problems. Helping students developing this ability is a focus of our
program, and this book --- \emph{Solving Real Problems with Data} --- is
an outgrowth of those efforts.

All too often, data science education is focused on the technical ---
coding, statistics, and model evaluation. Technical competence is
unquestionably important, but for applied data scientists interesting in
using their skills to solve problems (rather than just develop new
algorithms), it is just as important we prepare students to navigate the
ambiguity of real-world problems.

\emph{Solving Real Problems with Data} is designed to help fill this gap
in data science training. It provides readers with a systematic
framework for thinking about their goals and how to achieve them using
data science methods. It is written for data scientists who have already
learned the basics of statistical inference and machine learning, and
focuses instead on everything that comes before and after fitting a
model: how to work with stakeholders to clearly articulate the problem
they want to address; how to formulate questions whose answers will help
address your stakeholder's problem; how to choose the appropriate tool
based on the question one seeks to answer; and, critically, how to
evaluate and refine one's models based on your stakeholders needs.

With the help of this framework, as well as case studies and exercises,
\emph{Solving Real Problems with Data} will help data science students
develop a systematic understanding of how to approach and manage data
science projects from conception through delivery and adoption. It will
provide a unified perspective on how the perspectives and tools learned
in other courses complement one another, and when different approaches
to data science are most appropriate.

The need to help students develop these skills has never been higher.
Where once students could get good jobs on the basis of their ability to
code and implement data science tools, the rise of LLM-based coding
assistants is dramatically reducing the returns to a purely technical
education. To be competitive, students must bring more to the table than
an ability to code --- they must also be able to communicate,
collaborate, and bridge the gap between the technical and substantive.

\section*{2. The Market}\label{the-market}

This book is targeted at two markets: courses in applied data science
programs, and young data scientists interested in improving their
problem-solving skills. In that sense, it is similar to other books in
the data science space designed to help young analysts and data
scientists bridge the gap between the technical and applied, such as
\emph{Trustworthy Online Controlled Experiments} from Kohavi, Tang and
Xu (Cambridge University Press, 2020) or \emph{Regressions and Other
Stories} from Gelman, Hill and Vehtari (Cambridge University Press,
2020).

\subsection*{Applied Data Science
Programs}\label{applied-data-science-programs}

This book was developed to address a need in the curriculum of the
two-year, applied data science program with which I work (MIDS). As
such, it is well suited to use in any of the numerous applied data
science programs around the country, including (but not limited to)
programs like the Vanderbilt Masters in Data Science, the UNC Chapel
Hill Applied Data Science Masters, the Carnegie Mellon Master of Science
in Applied Data Science, the Columbia Data Science Institute MS in Data
Science, the University of Washington Data Science Masters, or the
Stanford Statistics and Data Science Masters.

This book would also be appropriate for the growing number of data
science Masters programs targeting specific substantive specializations,
such as the University of Chicago Masters in Computational Analysis and
Public Policy, or the new Columbia Masters in Political Analytics.

The book is also particularly well suited to the growing number of
programs that offer data science as an area of concentration within
existing programs, like Economics, Political Science, Statistics, and
Sociology. Many such programs are emerging in an effort to capture
student excitement about data science, but sometimes struggle to fully
integrate training in computational methods with substantive questions
of interest. This book's emphasis on developing critical reasoning and
problem-solving skills at precisely the place where the technical meets
the substantive makes it ideally suited for such programs.

Because the book assumes familiarity with the basics of statistical
inference and machine learning, it fits best in the middle or towards
the end of a Masters degree or undergraduate major. For example, it is
an excellent resource to pair with a capstone class, or a class on
project development.

\subsection*{Young Professionals}\label{young-professionals}

The book is also designed to be of interest to young data scientists in
industry. Because most assessments (exercises and exams) in data science
programs are focused on technical competence, and many data science
hiring processes are organized around technical interviews, students
frequently get the impression that technical skills are the only skills
required to be successful. As a result, it is only when students
graduate that they come to realize they are inadequately prepared to
effectively work with stakeholders or design and iterate problems.

\section*{3. About The Book}\label{about-the-book}

Few fields have shown as much promise to address the world's problems as
data science. Today, data science is improving our understanding of and
adaptation to climate change. It is being used in medicine to speed drug
discovery, improve the quality of X-rays and MRIs, and ensure that
patients receive appropriate medical care. It is used in courtrooms to
fight for fair elections and electoral maps and by data journalists to
document and communicate the injustices prevalent in our criminal
justice system and issues in policing.

Data science also enables new technologies that may improve our lives.
Autonomous drones are delivering blood and medical supplies to rural
health clinics from Rwanda to
\href{https://www.theverge.com/2020/5/27/21270351/zipline-drones-novant-health-medical-center-hospital-supplies-ppe/}{North
Carolina}, and driver-aid features continue to make progress in reducing
the over 30,000 traffic deaths and millions of injuries that occur in
the US alone every year. And nearly every facet of business --- from the
way businesses source materials and manage inventory to the way product
offerings respond to customer behavior --- has been reshaped by data
science.

At the same time, businesses and regulators are also coming to
appreciate the potential of data science tools to reinforce racial and
gender inequities. Algorithms at Amazon have been found to
\href{https://www.reuters.com/article/us-amazon-com-jobs-automation-insight/amazon-scraps-secret-ai-recruiting-tool-that-showed-bias-against-women-idUSKCN1MK08G}{discriminate
against female job applicants}. Medical algorithms have been found to
prioritize White patients over Black patients
\href{https://www.wired.com/story/how-algorithm-blocked-kidney-transplants-black-patients/}{for
kidney transplants} and
\href{https://www.washingtonpost.com/health/2019/10/24/racial-bias-medical-algorithm-favors-white-patients-over-sicker-black-patients/}{preventative
care}. In the criminal justice system, algorithms have been found to
\href{https://www.propublica.org/article/machine-bias-risk-assessments-in-criminal-sentencing}{incorrectly
identify Black defendants than White defendants as being a ``danger to
society'' when providing risk assessments to judges deciding on
pre-trial release, bail and sentencing}. And even Meta's own research
has shown its algorithms drive political polarization and division among
users, and push users into extremist groups.

How, then, should a burgeoning data scientist approach this discipline,
full of such promise and peril? Why have so many data science endeavors
failed to deliver on their promise? And why do we need \emph{yet
another} data science book?

\subsection*{This Book}\label{this-book}

This book is different from many other data science books you may have
read. Where most data science books are designed to teach specific data
science techniques or methods, the aim of this book is to provide you
with a framework for thinking about your goals and how to achieve them
using data science. It is, in a sense, about everything you need to know
\emph{beyond} the technicalities of model fitting. This is about
everything that comes \emph{before} and \emph{after} you fit your model:
it will help you work with stakeholders to clearly articulate the
problem they want to address, formulate questions whose answers will
help address your stakeholder's problem, choose an appropriate tool
based on the question you seek to answer, and, critically, evaluate and
refine your model based on your stakeholders needs.

The importance of these skills is often underestimated by data science
students, and for understandable reasons. Data science curricula usually
begin with coding, statistics, and model evaluation techniques. As a
result, the hardest part of data science classes is often mastering the
technical details of model implementation. Moreover, the limited time
available to instructors and the need to support full classes of
students means data science exercises almost always have to come with
clear directions and problem scaffolding to ensure students meet their
learning goals.

But real-world problems don't come with directions. Indeed, a problem
that is clearly defined and for which a solution is obvious isn't a
problem anyone will pay you very much to solve. No, classroom exercises
are carefully structured to foster learning and to make it possible for
instructors to grade and provide feedback at scale. But real problems
--- the kind you will encounter in industry, government, or research ---
are hard to even articulate clearly, never mind solve. And that is why,
as we will see, what really sets exceptional professional data
scientists apart is not their ability to get a high AUC --- \textbf{it's
their ability to navigate and thrive in the face of ambiguous problems
and goals.}

\subsection*{Four Big Ideas}\label{four-big-ideas}

This book is organized around four big ideas:

\begin{enumerate}
\def\labelenumi{\arabic{enumi}.}

\item
  \textbf{Data science is about solving problems.}
\end{enumerate}

All too often, young data scientists get lost in the technical details
of models and lose sight of the bigger picture. Data science is not
about maximizing accuracy --- it's about using data and quantitative
methods to solve problems, and at the end of the day the only ``metric''
that matters is whether your work has helped solve the problem you set
out to address.

To some readers, this idea may seem self-evident and uncontroversial. In
my experience as an instructor, however, this is not a natural
perspective for students. On one level they recognize that data science
is generally meant to accomplish a goal --- though when asked about the
goal of a data scientist they often offer more generic answers like
``generate insights from data'' or ``make recommendations,'' but then
struggle to explain what makes something an insight, or on what basis
one would make a recommendation. This lack of attention to project
motivation and design is often further reinforced by classroom projects
organized around instructor-provided, clearly articulated goals. And
even when students are left free to choose their own topics, those
topics are often backwards designed from available datasets and are
rarely the subject of faculty feedback.

That is not the case in this book. Time and again, students are pushed
to remember that difficult questions --- like how to weigh trade-offs in
loss-function specifications --- must always be answered \emph{in the
context of the problem you are trying to solve.}

The importance of problem articulation is also the subject of the first
two substantive chapters of the book. \textbf{Chapter 3: Solving the
Right Problem} addresses head-on the importance of clearly articulating
your problem in an actionable manner and illustrate through examples
what happens if you \emph{don't} articulate your problem correctly. It
provides concrete suggestions for how to reframe problems, and examples
of how reframing of problems has successfully helped simplify apparently
intractable problems in the past.

Then in \textbf{Chapter 4: Stakeholder Management}, the book steps back
from the unrealistic simplification that the problem is the
\emph{reader's} problem, and instead acknowledges that most data
scientists are working \emph{for} a stakeholder. This chapter discusses
the delicate art of working with stakeholders to refine your mutual
understanding of the problem you are trying to solve. Crucially, this
chapter discusses the importance of being respectful of the stakeholder
and their domain expertise while also not being \emph{overly}
deferential and assuming the stakeholder always knows best --- and
problem I often see with young data scientists.

\begin{enumerate}
\def\labelenumi{\arabic{enumi}.}
\setcounter{enumi}{1}

\item
  \textbf{Data scientists solve problems by answering questions.}
\end{enumerate}

The second big idea of the book is that data scientists solve problems
by answering questions about the world. Given that, we can reframe the
challenge of a data scientist from the more amorphous task of ``figuring
out how to solve the problem'' to the more concrete ``what question, if
answered, would make it easier to solve this problem?'' Moreover, once
we've articulated a question to answer, we can turn to choosing the best
tool for generating an answer to that specific type of question. And it
is only at this stage---not at the beginning!---that we start thinking
about what statistical method, algorithm, or model is most appropriate.

This idea emerges from the fact that all data science tools, at their
core, are tools for answering questions about the world, and that
understanding data science tools through the lens of ``what question is
this tool answering?'' is extremely powerful.

In the case of some tools --- like clustering algorithms or other
unsupervised machine learning tools --- this idea is relatively
uncontroversial. A clustering algorithms answer some form of the
question ``if I wanted to partition the observations in this data to
maximize the similarity of points within each cluster (in terms of a
specified set of variables and a similarity metric) and minimize
similarity between clusters, how would I do so?''

In other contexts --- like in supervised machine learning --- the
relevance of this perspective can seem less relevant (or potentially
just pedantic). But let's take the example of a supervised machine
learning algorithm designed to classify mammograms as normal or abnormal
trained on data labelled by radiologists at a Boston hospital, an
example I use in the book. What question is this model answering? It is
\emph{not} answering the question, for any input mammogram, ``Is this
mammogram normal or abnormal?'' Rather, in a very real sense it is
answering the question ``if this mammogram were shown to one of the
Boston radiologists how labelled the training data, how likely are
\emph{they} to label the mammogram as normal or abnormal?'' And while we
hope the answer to those questions is similar, being explicit about the
distinction makes clear how human biases and tendencies end up being
replicated in our models.

For experienced data scientists, this idea is, by this point, second
nature. But in my experience, reframing what predictive models are doing
in this manner to students very often causes a ``light bulb'' moment of
realization.

\begin{enumerate}
\def\labelenumi{\arabic{enumi}.}
\setcounter{enumi}{2}

\item
  \textbf{The questions data scientists answer can be divided into three
  categories: exploratory, passive predictive, and causal.}
\end{enumerate}

This third big idea is where this book departs more substantially from
just presenting existing ideas in what I hope is a particularly
effective manner and begins to lay out some new concepts. In particular,
the book presents students with a novel taxonomy of question types
organized around questions' substantive purposes, not intellectual
tradition from which they are drawn or the computational machinery being
deployed.

\textbf{Exploratory Questions} are questions about patterns and
regularities in the world around us. Answering Exploratory Questions
helps data scientists better understand the landscape of the problem
they wish to solve --- for example, by helping them understand where the
problem is most acute, or what factors seem most strongly correlated
with problem intensity --- and often aid in prioritizing subsequent
efforts.

Answering Exploratory Questions is largely the domain of statistical
inference and unsupervised machine learning. However, as I emphasize to
students, many of the tools and approaches to data science we encounter
are artifacts of how university departments are organized, and our
substantive goals as data scientists will not always follow the lines we
draw between computer science, statistics, and social science
departments.

Answering Exploratory Questions is \textbf{not} synonymous with
``Exploratory Data Analysis'' (EDA). As it is commonly understood and
practiced by students, EDA refers to the process of poking around in a
new data set before fitting a more complicated statistical model. It
entails learning what variables are present, how they are coded, and
\emph{sometimes} looking at general patterns in the data prior to model
fitting. Crucially, it is generally defined by what it is \emph{not} ---
it is what you do \emph{before} you fit a complicated model --- and by
the tools used --- EDA consists of plotting distributions or
cross-tabulations, not fitting models or doing more sophisticated
analyses.

Answering Exploratory Questions, by contrast, is about achieving a
substantive goal --- learning about patterns \emph{in the world} --- not
the tools used to do it, or what it comes before. Answering an important
Exploratory Question may require you to actively seek out new data,
merge data from different sources together, and potentially do novel
data collection. It may also entail model fitting or use of unsupervised
machine learning algorithms to uncover latent patterns.

Indeed, the book allocates a
\href{https://ds4humans.com/30_questions/07_eda.html}{full sub-chapter}
to discussion of the conceptual problems with how the term EDA is
commonly used.

\textbf{Passive Prediction Questions} are questions about the unobserved
outcomes of individual entities (people, stocks, stores, etc.). Because
Passive Prediction Questions are questions about individual entities,
they don't necessarily have one ``big'' answer. Rather, Passive
Prediction Questions are answered by fitting or training a model that
can take the characteristics of an individual entity as inputs (e.g.,
this patient is age 67, has blood pressure of 160/90, and no history of
heart disease) and spitting out an answer \emph{for that individual}
(given that, her probability of surgical complications is 82\%). This
differentiates Passive Prediction Questions from Exploratory Questions,
which are about global patterns, not individual level predictions.

Passive Prediction Questions are usually deployed for one of two
business purposes:

\begin{enumerate}
\def\labelenumi{\arabic{enumi})}

\item
  identifying individual entities of particular interest (high-risk
  patients, high-value clients, factory machinery in need of
  preventative maintenance, etc.), and
\item
  automating classification or labeling tasks currently performed by
  people (reading mammograms, reviewing job applicant resumes,
  identifying internet posts that violate terms of use).
\end{enumerate}

In general, these two business purposes correspond to the two types of
``predictions'' being made. Identifying individual entities of
particular interest is generally accomplished by creating predictions
about what future outcomes are likely to obtain absent intervention.
``Given this new customer's behavior on my website, are they likely to
spend a lot over the next year?''

This ability to make predictions about future outcomes is obviously of
tremendous use to stakeholders as it allows them to tailor their
approach at the individual level. A hospital that can predict which
patients are most likely to experience complications after surgery can
allocate their follow-up care resources accordingly. A business that
knows which customers are more likely to be big spenders can be sure
that those customers are given priority by customer care specialists.

But the meaning of the term ``Prediction'' in Passive Prediction
Questions extends beyond ``predicting the future''. Passive Prediction
Questions also encompass efforts to predict how a third party
\emph{would} behave or interpret something about an individual if given
the chance. Here we can return to the example of an ML model for reading
mammograms. Suppose a hospital stakeholder wanted to automate the
reading of mammograms so that rural hospitals without full-time
radiologists could give patients diagnoses more quickly (or, more
cynically, pay fewer radiologists). They could train a model by feeding
it a dataset of mammograms labelled by human radiologists. That model
would then effectively be answering the question: ``if a radiologist
looked at this particular scan, would they conclude it is abnormal?''

The value of this type of prediction to stakeholders is likely also
self-evident, as it opens the door for automation and scaling of tasks
that would otherwise be too costly or difficult for humans. Indeed,
answering this question is the type of task for which machine learning
has become most famous. Spam filtering amounts to answering the question
``If the user saw this email, would they tag it as spam?'' Automated
content moderation amounts to answering ``Would a Meta contractor
conclude the content of this photo violates Facebook's Community
Guidelines?'' Indeed, even Large Language Models (LLMs) like chatGPT,
Bard, and LLaMA can be understood in this way, as we will discuss later.

Again, this framing of what supervised machine learning models do as
answering this type of question may feel slightly pedantic to some
readers, but in my experience it helps students understand how these
types of models inherit the limitations of their training data, and open
the door to discussion of issues of misalignment which I also address.

Finally, \textbf{Causal Questions} are questions about the likely
consequences of actions the stakeholder. Causal Questions arise when
stakeholders want to \emph{do} something --- buy a Superbowl ad, change
how the recommendation engine in their app works, authorize a new
prescription drug --- but they fear the action they are considering may
be costly and not actually work. In these situations, stakeholders will
often turn to a data scientist in the hope that the scientist can
provide greater certainty about the likely consequences of different
courses of action before the stakeholder is forced to act at scale.
This, in turn, helps to reduce the risk the stakeholder has to bear when
making their decision --- something all stakeholders appreciate.

By emphasizing that Causal Questions in an applied data scientist's
career are generally motivated by a desire to understand an action the
stakeholder is considering taking, this book is able to discuss concepts
of external validity much more concretely than in many other texts on
causal inference which --- in my experience --- tend to be motivated by
social science questions in which there is no clear, specific
intervention in mind for which external validity becomes a first-order
concern.

\begin{enumerate}
\def\labelenumi{\arabic{enumi}.}
\setcounter{enumi}{3}

\item
  \textbf{Reasoning rigorously about uncertainty and errors is what
  differentiates good data scientists from great data scientists.} Data
  science isn't just about minimizing classification errors and
  uncertainty --- it's also about deciding how unavoidable errors should
  be distributed, how to weigh the risks and trade-offs inherent in
  probabilistic decision-making rigorously and in a manner that takes
  into account the problem you are trying to solve, and to take
  uncertainty into account when acting on data.
\end{enumerate}

Here again this books emphasis on starting with a motivating problem
aids dramatically in its ability to keep discussion of errors and
trade-offs when making decisions under uncertainty concrete. Whether in
discussion of the problem with p-values, the distinction between
substantive and statistical significance, or in discussing custom loss
functions when doing classification, a continual emphasis on how to
think about these issues \emph{in the context of the problem you are
seeking to solve} helps keep things concrete.

\subsection*{Table of Contents}\label{table-of-contents}

\begin{itemize}

\item
  \textbf{Chapter 1: Introduction:} An overview of the philosophy of the
  book and an introduction to the taxonomy of questions used throughout
  the book.
\end{itemize}

\textbf{Part 1: Solving Problems}

\emph{Part 1 details the importance of identifying and properly
articulating the problem one wishes to solve, as well as how to refine
one's understanding of their problem by working with your stakeholder.}

\begin{itemize}

\item
  \textbf{Chapter 2: Solving the \emph{Right} Problem: The Importance of
  Problem Articulation:} When starting a new project, there is always a
  temptation to dive into the data. After all, few things in data
  science feel quite as satisfying as successfully fitting a model. But
  when you fail to stop and ensure you clearly understand the problem
  you are trying to solve, it is easy to spend large amounts of energy
  creating technically impressive results that, in the end, in no way
  solve a problem or improve the lives of anyone involved.
\item
  \textbf{Chapter 3: Stakeholder Management:} Most data scientists are
  not actually employed to solve their own problems; they are employed
  to help solve the problem of a stakeholder. The chapter discusses the
  art of ``stakeholder management'' --- working \emph{with} a
  stakeholder to create a clear problem statement and plan.
\end{itemize}

\textbf{Part 2: Solving Problems Using Questions}

\emph{Part 2 details how data scientists use different types of data
science questions to solve problems.}

\begin{itemize}

\item
  \textbf{Chapter 5: Descriptive v. Prescriptive Questions:} Before
  discussing each of the three types of questions introduced in the
  introduction in detail, this chapter takes one step back to introduce
  the distinction between descriptive questions (questions about how the
  world \emph{is}) and prescriptive questions (questions about how the
  world \emph{should} be). The main goal of the chapter is to ensure
  data scientists recognize when the questions they are answering entail
  value judgments.
\item
  \textbf{Chapter 6: Using Exploratory Questions:} Often the most
  underappreciated type of question, Exploratory Questions play a
  critical role in helping data scientists ensure they putting their
  energy in the right places. Exploratory Questions help data scientists
  better understand the contours of the problem they wish to solve, such
  as where their problem is most acute (and thus where energy is likely
  to be best spent addressing the problem), or where the problem may
  have already been solved (and thus where they may look for
  inspiration). This chapter also includes an
  \href{https://ds4humans.com/30_questions/07_eda.html}{extended
  discussion} on the distinction between the practice commonly referred
  to as ``Exploratory Data Analysis (EDA)'' and what is meant by
  Exploratory Questions.
\item
  \textbf{Chapter 7: Using Passive Predictive Questions:} Passive
  Predictive Questions are the trendiest type of question in data
  science today. This chapter discusses the two major ways Passive
  Predictive Questions are used in business --- identifying individual
  entities for additional attention and automation. It also dives deeply
  into the distinction between Exploratory Questions and Passive
  Predictive Questions. Students often struggle with the idea that the
  same tool --- say, logistic regression --- can be used to answer
  different types of questions, but that the \emph{way} we use and
  evaluate model performance changes depending on our goals. Discussing
  how what matters most when answering Passive Prediction Questions
  (e.g., classification accuracy) may be different from what matters
  most when answering Exploratory Questions (e.g., the size of standard
  errors on regressors) helps students begin to recognize that there is
  no ``one way'' to use data science tools correctly; the correct way to
  use a tool depends on ones substantive goals.
\item
  \textbf{Chapter 8: Using Causal Questions:} Causal Questions are some
  of the most powerful and yet difficult questions in data science to
  answer. Causal Questions most often arise when a stakeholder is
  interested in undertaking a high-stakes action, but wishes to de-risk
  that decision by modelling its likely outcome. This chapter discusses
  the nature of Causal Questions and what differentiates them from
  Passive Predictive Questions. It also then touches on the distinction
  between experimental and observational methods and when each is more
  appropriate.
\end{itemize}

\textbf{Part 3: Reasoning About Uncertainty and Errors}

\emph{Part 3 is the meatiest portion of the book, and discusses the
types of issues that arise when answering different types of questions.
The emphasis of this Part is on the types of issues that tend not to be
emphasized in introductory statistics or machine learning courses. This
Part will therefore skip over topics like overfitting and model
diagnostics, and instead focus on concepts like external validity,
Goodhart's Law/Campbell's Law/The Lucas Critique, adversarial users,
adverse selection in deployment, statistical decision-making,
customizing loss functions to suit the substantive context, and the role
of ethics in loss-functions.}

\begin{itemize}

\item
  \textbf{Chapter 9: Internal and External Validity:} This chapter
  revisits what is hopefully (but is not always) a familiar concept for
  students: the distinction between \emph{internal validity} (how well a
  model or analysis answers a question for the population being
  analyzed) and \emph{external validity} (how well a model or analysis
  can be expected to generalize to a specific alternative population or
  context). It discusses why internal validity is often more emphasized
  in classes, and the fact a study's internal validity is a relatively
  singular property, while external validity can only be defined with
  respect to the specific context to which one wishes to generalize
  results.
\item
  \textbf{Chapter 10: Exploratory Question and Faithful Summarizations:}
  Answering Exploratory Questions can, at times, feel like the easiest
  part of data science. But whether one is calculating summary
  statistics or applying unsupervised machine learning models, answering
  Exploratory Questions requires the data scientist to exercise
  tremendous discretion and judgment. And when that discretion is not
  exercised thoughtfully, things can easily go awry. Answering
  Exploratory Questions boils down to generating faithful,
  understandable summaries of meaningful patterns in the data. Yet while
  we have lots of tools that generate summaries of our data, determining
  whether those summaries are faithful representations of the data, and
  whether the patterns they identify are meaningful depends --- and I
  know I'm starting to sound like a broken record here --- on the
  problem one is trying to solve. Students often fall into the trap of
  thinking tools like clustering algorithms or dimensionality reduction
  tools magically find the ``best'' summaries in the data. Moreover,
  they can be seduced by the fact that these tools will always return a
  result --- generally the best result for a given loss function. But
  just because something is the \emph{best} way to, say, cluster the
  data doesn't mean that those clusters are actually that distinct.
  Moreover, the clusters that minimize the default loss function from a
  clustering package may not be the clusters that are most relevant for
  a given problem. A Netflix engineer interested in creating interest
  clusters may find that their clustering algorithm finds the cleanest
  clusters occur at the level of broad genres --- horror, comedy,
  documentaries, etc. But if the engineer's goal is to be able to make
  suggests that had not occurred to a viewer, then smaller clusters ---
  say, romantic comedies with same-sex protagonists, and science fiction
  films with existential themes --- may be far more useful.
\item
  \textbf{Chapter 11: Passive Predictive Questions and The Right Way to
  be Wrong (Internal Validity):} Whether as a consequence of
  over-exposure to Kaggle competitions or a sense that models are meant
  to be evaluated on an absolute scale analogous to academic grades,
  many young data scientists become fixated on generic measures of model
  quality like accuracy or AUC. This chapter dives into the problems
  with myopic evaluation of models using generic metrics. It discusses
  the importance of benchmarking to alternatives, and the importance of
  customizing loss functions to suit a specific substantive problem.
  False positives and false negatives have different substantive
  importance in different contexts --- when designing a cancer detection
  algorithm, a false negative is far more consequential than a false
  positive, but when issuing credit, a false positive is likely to be
  more financially consequential than a false negative. The chapter also
  discusses the use of machine learning in the criminal justice system,
  and dives into how the choice of how to balance different types of
  errors is not just a technical question, it is also an ethical one.
  \textbackslash{}
\item
  \textbf{Chapter 12: Passive Predictive Questions and Strategic Actors
  (External Validity):} This chapter discusses the many problems that
  arise when a model trained on historic data is deployed in the real
  world and the people on the other side of that model respond
  strategically. The problem of people responding to models
  strategically goes by many names --- Goodhart's Law, Campbell's Law,
  The Lucas Critique, adversarial users, and adverse selection in
  deployment --- but underlying all these terms is the same problem:
  deploying a model changes the incentives people face, and people
  respond to incentives. From Zillow's disastrous entry into iBuying to
  students gaming robograders, and from fraud detection to the search
  engine optimization (SEO), this chapter dives into a range of examples
  of the ways in which fitting a model is only the first step in using
  predictive modelling successfully.\\
\item
  \textbf{Chapter 13: Causal Questions and The Challenges of
  Deployment:} One of the hardest parts of answering Causal Questions is
  that data scientists are called in to answer them in contexts where
  the stakes are too high to try things at a large scale. This chapter
  focuses on the issues of external validity --- such as general
  equilibrium effects, novelty and primacy effects, etc. --- that arise
  when one attempts to generalize the results of causal analyses. This
  section is of particular interest to data scientists, many of whom
  have the privilege of working at companies with A/B testing
  infrastructure that radically simplifies the challenge of ensuring
  high internal validity, and who are likely to have had internal
  validity emphasized in courses.
\item
  \textbf{Chapter 14: Causal Questions and Statistical Decision-Making:}
  Far, far too many students confuse statistical significance with
  substantive significance, and assume that statistical significance is
  determined by absolute cutpoints (p\textless0.05). This chapter takes
  a deep dive into these concepts with particular attention the meaning,
  interpretation, and limitations of p-values (especially in the context
  of experiments). This emphasis on p-values, to be clear, is not
  motivated for any particular love of the statistic, but rather an
  acknowledgement of their pervasive use in academia and industry. This
  chapter will also introduce how Bayesian statistics offers an
  alternative and, in the view of many, more intellectually coherent
  approach to statistically decision-making. This introduction is mean
  to ensure students are \emph{aware} of alternatives to the frequentist
  approach to inference, while not trying to cover it in detail.
\end{itemize}

\textbf{Part 4: Data Science Professionalization}

\begin{itemize}

\item
  \textbf{Chapter 15: What \emph{is} Data Science? A Historical
  Perspective:} Data science has yet to reach the level of maturation of
  fields like computer science, economics, or mechanical engineering.
  One consequence of this immaturity is that students often struggle to
  understand when differences in terminology are substantive and when
  they simply represent different intellectual traditions coming up with
  different names for the same phenomenon. In this chapter, I discuss
  the ways in which the organization of universities has shaped the
  sometimes myopic exposure to the field many practicing data scientists
  received in school and the fragmented language around data science.
  This, I argue, is important for young data scientists to know as it is
  both the source of some of the most exciting opportunities for
  intellectual arbitrage across bureaucratic divisions, and also some of
  the biggest challenges a holistically trained data scientist is likely
  to face when working with data scientists trained in a more
  traditional academic silo.
\item
  \textbf{Chapter 16: Writing to Stakeholders:} This chapter details
  principles for communicating with stakeholders. The main goal of this
  chapter is to help students understand the difference between the
  types of reports they have written in classes and reports and memos
  written for stakeholders. Reports written for classes often follow an
  academic structure (introduction, lit review, methods, results,
  conclusions), and tend to include extensive details about all the work
  students completed to aid instructors who want to check the student
  did their work correctly. Reports for stakeholders, by contrast, are
  about communicating a conclusion to a time-constrained reader. This
  chapter introduces the ``inverted pyramid'' organization structure
  from journalism (lead with what matters most, then double back and
  fill in details), and emphasizes the importance of being succinct and
  including only what is relevant to the take-away one wishes to
  communicate.
\item
  \textbf{Chapter 17: Providing Feedback:} But helping your colleagues
  be more effective isn't just a good thing to do from an ethical
  perspective --- it will also make your teams and units more effective,
  improve your professional reputation, make people eager to involve you
  in new projects, and make colleagues more likely to reciprocate with
  your generosity. Providing detailed, constructive feedback is one of
  the best ways to help colleagues. Yet it's a skill we rarely teach. In
  this reading, I provide a framework for giving feedback that's
  specific to empirical data science. As with everything in this book,
  you may eventually decide this approach to feedback is not for you,
  and that you already have a style of feedback you enjoy. But give it a
  try and if nothing else, it will give you a foil to help you
  understand what you \emph{don't} like in feedback!
\end{itemize}

\section*{4. Competition}\label{competition}

There are \emph{many} books in the data science space, but relatively
few that I see filling the niche I hope to serve. For example, O'Reilly
Publishing has published many books on the topic of data science, but
nearly all emphasize coding and modelling, not critical thinking at the
interface of the technical and substantive.

The closest peer in terms of \emph{content} is probably \emph{The Art of
Data Science} by Peng and Matsui (Lulu.com Publishing, 2016). Though
short, the book aims to provide readers with a similar holistic
perspective on the lifecycle of a data science project. To be entirely
honest, I have avoided reading it \emph{too} closely out of a desire to
avoid letting it color my writing, but there are several distinctions
between its approach and my own.

First, it does not assume prior familiarity with statistical inference
or machine learning. As a result, it is forced to operate at a more
superficial level of analysis, and spends far more ink explaining basic
concepts. Second, it was written in 2016, and as a result it is unable
to engage with many of the more recent technical developments in data
science (e.g., LLMs) or more recent case studies around topics like
algorithmic bias.

\section*{5. Production Basics}\label{production-basics}

At this time, I have completed a draft of the majority of the book
manuscript. I have also had the opportunity to use the material in the
book in class with MIDS students several times, allowing for refinement
of the material. The current draft compiles to about 140 pages A4, and I
imagine a final draft would be of similar length. The text has a fair
number of figures, but I do not see color printing being a requirement.
It seems like its most natural format would be as an A5 sized book.

In addition to publication of a physical book, I also this the book
living online in a manner similar to an increasing number of books, like
\href{https://mixtape.scunning.com/}{Scott Cunningham's \emph{Causal
Mixtape}} or \href{https://avehtari.github.io/ROS-Examples/}{Gelman,
Hill, and Vehtari's \emph{Regression and Other Stories}}. Indeed, the
project has mostly been developed in an online format, as can be seen at
\url{https://ds4humans.com}. I have primarily written the material in
Markdown and compiled it to HTML using
\href{https://jupyterbook.org/en/stable/intro.html}{Jupyter Book}, a
system with also supports compilation to LaTeX. The current projects
materials can be found
\href{https://github.com/nickeubank/ds4humans}{here}, and a PDF build
can be found
\href{https://github.com/nickeubank/ds4humans/blob/main/_build/latex/ds4humans.pdf}{here}.

My hope is to work on tightening up the manuscript this summer,
including filling in some missing sections, then use the refined
manuscript in my spring course one last time for refinement before
calling it ``done.'' With that in my, my goal for completion would be
early summer 2026.

\section*{6. Open Access}\label{open-access}

I am very interested in making the materials in this book open access,
although I have not yet secured funding specifically for that purpose.

\section*{7. Supplementary Materials}\label{supplementary-materials}

Because I have been using this material in my course, I have also been
developing exercises to go along with it, including in-class exercises,
coding exercises, and project outlines. These are all materials I look
forward to providing alongside the text.

\section*{8. About The Author}\label{about-the-author}

The interdisciplinary and eclectic nature of the book reflects my own
background. As an undergraduate, I studied Economics and International
Relations at Pomona College, then set out for what I thought was the
start of a career as an empirical development economist by joining the
Development Economic Research Group at the World Bank. After a few years
analyzing data on the educational ecosystem in India and Pakistan,
however, I became disillusioned with the approach of macroeconomics and
turned my interest to the intersection of political science and
economics. I did some additional work on the topic at the World Bank and
with the Center for Global Development in Washington, DC, then started a
PhD at Stanford. While completing my PhD in Political Economy from the
Stanford Graduate School of Business and an MA in Economics, I
discovered the power of data science to answer the questions I cared
about and dove deeply into computer science and computer engineering.

Around the time I completed my PhD, the focus of my work shifted away
from international development issues for personal reasons. Instead, I
turned my attention --- and newfound data science skills --- towards the
US electoral system. Together with Jonathan Rodden, I developed new
techniques for measuring the fairness of electoral districts and began
to supplement my academic publishing with working as an Expert
Consultant on voting rights and gerrymandering litigation.

Today, I am an Assistant Research Professor in Political Science at Duke
University, the Faculty Director of the Masters of Interdisciplinary
Data Science (MIDS) program, and Associate Director for the Rhodes
Information Initiative @ Duke. My research has been published in top
political science journals --- including the \emph{American Political
Science Review}, \emph{Political Analysis}, \emph{Quarterly Journal of
Political Science}, and \emph{Political Science Research and Methods}.
My work has been covered by \emph{The Economist} and the
\emph{Washington Post}, and in an Op-Ed in \emph{The Guardian}. And
perhaps most importantly I have had the privilege to have worked on
voting rights cases in Kansas, Arizona, Ohio, and North Carolina.

%------------------------------ END OF DOCUMENT ------------------------------------%

\end{document}
