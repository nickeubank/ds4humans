

\documentclass[12pt]{article}


\usepackage[T1]{fontenc}
\usepackage{amsfonts, amsmath, amssymb}
\usepackage{authblk}
\usepackage{multirow}
\usepackage{epsfig}
\usepackage{subfigure}
\usepackage{subfloat}
\usepackage{graphicx}
\usepackage{lscape}
\usepackage{amsmath}

\usepackage{amssymb}
\usepackage{gensymb}
\usepackage{tabularx}

\usepackage{booktabs}
\usepackage{longtable}
\usepackage{verbatim, rotating, paralist}
\usepackage{enumerate}
\usepackage{natbib}
\usepackage{multibib}
\usepackage{pdfsync}
\usepackage{latexsym}
\usepackage{amsthm}


\usepackage{stmaryrd}
\usepackage{dsfont}
\usepackage{hyperref}
\usepackage{bbm}
\usepackage{mathtools}

\usepackage{parskip}
\usepackage{anysize, indentfirst, setspace}
\usepackage[right=2.1cm, left=2.1cm, top=2.25cm, bottom=2.25cm]{geometry}
\usepackage{tikz}
\usepackage{epigraph}
\usepackage{csquotes}
\usepackage{appendix}
\usepackage{enumitem}
\usepackage{rotating}
\usepackage{changepage}


\renewcommand{\topfraction}{.85}
\renewcommand{\bottomfraction}{.7}
\renewcommand{\textfraction}{.15}
\renewcommand{\floatpagefraction}{.66}
\renewcommand{\dbltopfraction}{.66}
\renewcommand{\dblfloatpagefraction}{.66}

\usetikzlibrary{arrows,shapes,backgrounds,positioning,patterns,decorations.pathreplacing}
\newenvironment{shift}{\begin{adjustwidth}{1cm}{}}{\end{adjustwidth}}
\setlist{nosep}
\newcites{sec}{Appendix References}

%\renewcommand{\footnotelayout}{\doublespacing}
%\linespread{1.5}

\title{Solving Real Problems with Data \\ An Intermediate Data Science Text}

%-----------------------------------BEGIN DOCUMENT--------------------------------%
\begin{document}

\setlength{\parindent}{0.0in}
\setlength{\parskip}{.125in}


\date{\today }
%\date{April 25, 2021 }
%\date{November 24, 2020 }


\maketitle

\section*{1. The 30-Second Sell}\label{the-30-second-sell}

In my six years working with students in the Duke Masters of
Interdisciplinary Data Science (MIDS) program --- of which I am now the
Faculty Director --- I have found that the hardest part of training data
scientists preparing them for the transition from doing technically
challenging but well-structured classroom exercises to applying data
science tools effectively when faced with the ambiguities of real world
problems. Helping students developing this ability is a focus of our
program, and this book --- \emph{Solving Real Problems with Data} --- is
an outgrowth of those efforts.

All too often, data science education is focused on the technical ---
coding, statistics, and model evaluation. Technical competence is
unquestionably important, but for applied data scientists interesting in
using their skills to solve problems (rather than just develop new
algorithms), it is just as important we prepare students to navigate the
ambiguity of real-world problems.

\emph{Solving Real Problems with Data} is designed to help fill this gap
in data science training. It provides readers with a systematic
framework for thinking about their goals and how to achieve them using
data science methods. It is written for data scientists who have already
learned the basics of statistical inference and machine learning, and
focuses instead on everything that comes before and after fitting a
model: how to work with stakeholders to clearly articulate the problem
they want to address; how to formulate questions whose answers will help
address your stakeholder's problem; how to choose the appropriate tool
based on the question one seeks to answer; and critically how to
evaluate and refine one's models based on your stakeholders needs.

With the help of this framework, as well as case studies and exercises,
\emph{Solving Real Problems with Data} will help data science students
develop a systematic understanding of how to approach and manage data
science projects from conception through delivery and adoption. It will
provide a unified perspective on how the perspectives and tools learned
in other courses complement one another, and when different approaches
to data science are most appropriate.

\section*{2. The Market}\label{the-market}

This book is targeted at two markets: courses in applied data science
programs, and young data scientists interested in improving their
problem-solving skills. In that sense, it is similar to other books in
the data science space designed to help young analysts and data
scientists bridge the gap between the technical and applied, such as
\emph{Trustworthy Online Controlled Experiments} from Kohavi, Tang and
Xu (Cambridge University Press, 2020) or \emph{Regressions and Other
Stories} from Gelman, Hill and Vehtari (Cambridge University Press,
2020).

\subsection*{Applied Data Science
Programs}\label{applied-data-science-programs}

This book was developed to address a need in the curriculum of the
two-year, applied data science program with which I work (MIDS). As
such, it is well suited to use in any of the numerous applied data
science programs around the country, including (but not limited to)
programs like the Vanderbilt Masters in Data Science, the UNC Chapel
Hill Applied Data Science Masters, the Carnegie Mellon Master of Science
in Applied Data Science, the Columbia Data Science Institute MS in Data
Science, the University of Washington Data Science Masters, or the
Stanford Statistics and Data Science Masters.

This book would also be appropriate for the growing number of data
science Masters programs targeting specific substantive specializations,
such as the University of Chicago Masters in Computational Analysis and
Public Policy, or the new Columbia Masters in Political Analytics.

The book is also particularly well suited to the growing number of
programs that offer data science as an area of concentration within
existing programs, like Economics, Political Science, Statistics, and
Sociology. Many such programs are emerging in an effort to capture
student excitement about data science, but sometimes struggle to fully
integrate training in computational methods with substantive questions
of interest. This book's emphasis on developing critical reasoning and
problem-solving skills at precisely the place where the technical meets
the substantive makes it ideally suited for such programs.

Because the book assumes familiarity with the basics of statistical
inference and machine learning, it fits best in the middle or towards
the end of a Masters degree or undergraduate major. For example, it is
an excellent resource to pair with a capstone class, or a class on
project development.

\subsection*{Young Professionals}\label{young-professionals}

The book is also designed to be of interest to young data scientists in
industry. Because most assessments (exercises and exams) in data science
programs are focused on technical competence, and many data science
hiring processes are organized around technical interviews, students
frequently get the impression that technical skills are the only skills
required to be successful. As a result, it is only when students
graduate that they come to realize they are inadequately prepared to
effectively work with stakeholders or design and iterate problems.

\section*{3. About The Book}\label{about-the-book}

\section*{4. Competition}\label{competition}

There are \emph{many} books in the data science space, but relatively
few that I see filling the niche I hope to serve. For example, O'Reilly
Publishing has published many books on the topic of data science, but
nearly all emphasize coding and modelling, not critical thinking at the
interface of the technical and substantive.

The closest peer in terms of \emph{content} is probably \emph{The Art of
Data Science} by Peng and Matsui (Lulu.com Publishing, 2016). Though
short, the book aims to provide readers with a similar holistic
perspective on the lifecycle of a data science project. To be entirely
honest, I have avoided reading it \emph{too} closely out of a desire to
avoid letting it color my writing, but there are several distinctions
between its approach and my own.

First, it does not assume prior familiarity with statistical inference
or machine learning. As a result, it is forced to operate at a more
superficial level of analysis, and spends far more ink explaining basic
concepts. Second, it was written in 2016, and as a result it is unable
to engage with many of the more recent technical developments in data
science (e.g., LLMs) or more recent case studies around topics like
algorithmic bias.

\section*{5. Production Basics}\label{production-basics}

At this time, I have completed a draft of the majority of the book
manuscript. I have also had the opportunity to use the material in the
book in class with MIDS students several times, allowing for refinement
of the material. The current draft compiles to about 140 pages A4, and I
imagine a final draft would be of similar length. The text has a fair
number of figures, but I do not see color printing being a requirement.
It seems like its most natural format would be as an A5 sized book.

In addition to publication of a physical book, I also this the book
living online in a manner similar to an increasing number of books, like
\href{https://mixtape.scunning.com/}{Scott Cunningham's \emph{Causal
Mixtape}} or \href{https://avehtari.github.io/ROS-Examples/}{Gelman,
Hill, and Vehtari's \emph{Regression and Other Stories}}. Indeed, the
project has mostly been developed in an online format, as can be seen at
\url{https://ds4humans.com}. I have primarily written the material in
Markdown and compiled it to HTML using
\href{https://jupyterbook.org/en/stable/intro.html}{Jupyter Book}, a
system with also supports compilation to LaTeX. The current projects
materials can be found
\href{https://github.com/nickeubank/ds4humans}{here}, and a PDF build
can be found
\href{https://github.com/nickeubank/ds4humans/blob/main/_build/latex/ds4humans.pdf}{here}.

My hope is to work on tightening up the manuscript this summer,
including filling in some missing sections, then use the refined
manuscript in my spring course one last time for refinement before
calling it ``done.'' With that in my, my goal for completion would be
early summer 2026.

\section*{6. Open Access}\label{open-access}

I am very interested in making the materials in this book open access,
although I have not yet secured funding specifically for that purpose.

\section*{7. Supplementary Materials}\label{supplementary-materials}

Because I have been using this material in my course, I have also been
developing exercises to go along with it, including in-class exercises,
coding exercises, and project outlines. These are all materials I look
forward to providing alongside the text.

\section*{8. About The Author}\label{about-the-author}

The interdisciplinary and eclectic nature of the book reflects my own
background. As an undergraduate, I studied Economics and International
Relations at Pomona College, then set out for what I thought was the
start of a career as an empirical development economist by joining the
Development Economic Research Group at the World Bank. After a few years
analyzing data on the educational ecosystem in India and Pakistan,
however, I became disillusioned with the approach of macroeconomics and
turned my interest to the intersection of political science and
economics. I did some additional work on the topic at the World Bank and
with the Center for Global Development in Washington, DC, then started a
PhD at Stanford. While completing my PhD in Political Economy from the
Stanford Graduate School of Business and an MA in Economics, I
discovered the power of data science to answer the questions I cared
about and dove deeply into computer science and computer engineering.

Around the time I completed my PhD, the focus of my work shifted away
from international development issues for personal reasons. Instead, I
turned my attention --- and newfound data science skills --- towards the
US electoral system. Together with Jonathan Rodden, I developed new
techniques for measuring the fairness of electoral districts and began
to supplement my academic publishing with working as an Expert
Consultant on voting rights and gerrymandering litigation.

Today, I am an Assistant Research Professor in Political Science at Duke
University, the Faculty Director of the Masters of Interdisciplinary
Data Science (MIDS) program, and Associate Director for the Rhodes
Information Initiative @ Duke. My research has been published in top
political science journals --- including the \emph{American Political
Science Review}, \emph{Political Analysis}, \emph{Quarterly Journal of
Political Science}, and \emph{Political Science Research and Methods}.
My work has been covered by \emph{The Economist} and the
\emph{Washington Post}, and in an Op-Ed in \emph{The Guardian}. And
perhaps most importantly I have had the privilege to have worked on
voting rights cases in Kansas, Arizona, Ohio, and North Carolina.

%------------------------------ END OF DOCUMENT ------------------------------------%

\end{document}
