% !TEX program = lualatex

\documentclass[12pt]{beamer}
% For handout add ,handout after 11pt

\usetheme{auriga}
\usecolortheme{auriga}

\usepackage{booktabs}
\usepackage{graphicx}
\usepackage[export]{adjustbox}
\usepackage{color}
\usepackage{tikz}

% Set up emoji support with a fallback font
\newfontfamily\emojifont{Apple Color Emoji}[Renderer=Harfbuzz]
\DeclareTextFontCommand{\emoji}{\emojifont}

% Beamer list spacing
% \setlength{\leftmargini}{1.5em}
\setlength{\topsep}{1em}
\setlength{\itemsep}{1em}
\setlength{\parskip}{1.5em}
\setlength{\parsep}{0.5em}

% define some colors for a consistent theme across slides
\definecolor{red}{RGB}{181, 23, 0}
\definecolor{blue}{RGB}{0, 118, 186}
\definecolor{gray}{RGB}{146, 146, 146}

\title{Functional Forms \& Matching}

\author{Nick Eubank}

\date{\vspace*{.3in} \date}

\begin{document}
\setbeamercolor{page number in head/foot}{fg=white}

\begin{frame}[c]{}
    \titlepage
\end{frame}

\begin{frame}
    \frametitle{Functional Forms}
\pause You work for a mortgage lender building a credit-risk model.\\[1ex]
You are analyzing historic mortgages, regressing delinquency on mortgage features. \\[1ex]
\pause You fit the following model:
\begin{eqnarray}
    delinquency_i = \beta_0 &+& \beta_1 credit\_score_i + \beta_2 income_i \nonumber \\
    & +&  \beta_3 interest\_rate_i + \epsilon_i \nonumber
\end{eqnarray}
What \alert{functional form assumptions} are implicit in this specification?
\end{frame}


\begin{frame}
    \frametitle{Functional Forms}
\begin{eqnarray}
    delinquency_i = \beta_0 &+& \beta_1 credit\_score_i + \beta_2 income_i \nonumber \\
    & +&  \beta_3 interest\_rate_i + \epsilon_i \nonumber
\end{eqnarray}
\pause Question 1:
\begin{itemize}
    \item Borrower A's income rises from \$20,000 to \$30,000
    \item Borrower B's income rises from \$100,000 to \$110,000
\end{itemize}
\pause Does Borrower A's delinquency risk decrease \alert{by more, less or the same amount} as Borrower B's risk?
\end{frame}

\begin{frame}
    \frametitle{Functional Forms}
\begin{eqnarray}
    delinquency_i = \beta_0 &+& \beta_1 credit\_score_i + \beta_2 ln(income)_i \nonumber \\
    & +&  \beta_3 interest\_rate_i + \epsilon_i \nonumber
\end{eqnarray}
\pause Question 2:
\begin{itemize}
    \item Borrower A's income rises from \$20,000 to \$30,000
    \item Borrower B's income rises from \$100,000 to \$110,000
\end{itemize}
\pause Does Borrower A's delinquency risk decrease \alert{by more, less or the same amount} as Borrower B's risk?
\end{frame}


\begin{frame}
    \frametitle{Functional Forms}
\begin{eqnarray}
    delinquency_i = \beta_0 &+& \beta_1 credit\_score_i + \beta_2 income_i \nonumber \\
    & +&  \beta_3 interest\_rate_i + \epsilon_i \nonumber
\end{eqnarray}
\pause Question 3:
\begin{itemize}
    \item Borrower A has an income of \$50,000.
    \item Borrower B has an income of \$500,000.
\end{itemize}
\pause Interest rates rise by 1 percentage point.\\[1ex]
\pause Does Borrower A's delinquency risk decrease \alert{by more, less or the same amount} as Borrower B's risk?
\end{frame}

\begin{frame}
    \frametitle{Matching \& Functional Forms}
The promise of Matching is that it helps solve these problems. \\[1ex]
\pause So is it a \alert{silver bullet} for letting the data determine how our features impact outcomes?
\end{frame}

\begin{frame}
    \frametitle{}

    \begin{center}
    \begin{tikzpicture}[scale=1.5]
        % Draw grey grid lines
        \draw[gray, thin] (0,0) grid (3,3);
        
        % Draw axes
        \draw[->] (0,0) -- (3.5,0) node[right] {$x$};
        \draw[->] (0,0) -- (0,3.5) node[above] {$y$};
        
        % Add tick marks and labels for x-axis
        \foreach \x in {0,1,2,3}
            \draw (\x,0.05) -- (\x,-0.05) node[below] {\x_1};
        
        % Add tick marks and labels for y-axis
        \foreach \y in {0,1,2,3}
            \draw (0.05,\y) -- (-0.05,\y) node[left] {\x_2};
        
        % Add points
        \filldraw[black] (1,1) circle (2pt);
        \filldraw[blue] (2,2) circle (2pt) node[above right=2pt] {Obs 1: (2,2)};
        \filldraw[red] (3,1.5) circle (2pt) node[right=2pt] {Obs 2: (3, 1.5)};
    \end{tikzpicture}
    \end{center}
\pause Which point should be matched with the Black point? Why?
\end{frame}


\begin{frame}
    \frametitle{}

    \begin{center}
    \begin{tikzpicture}[scale=1.5]
        % Draw grey grid lines
        \draw[gray, thin] (0,0) grid (3,3);
        
        % Draw axes
        \draw[->] (0,0) -- (3.5,0) node[right] {$x$};
        \draw[->] (0,0) -- (0,3.5) node[above] {$y$};
        
        % Add tick marks and labels for x-axis
        \foreach \x in {0,1,2,3}
            \draw (\x,0.05) -- (\x,-0.05) node[below] {\x_1};
        
        % Add tick marks and labels for y-axis
        \foreach \y in {0,1,2,3}
            \draw (0.05,\y) -- (-0.05,\y) node[left] {\x_2};
        
        % Add points
        \filldraw[black] (1,1) circle (2pt);
        \filldraw[blue] (2,2) circle (2pt) node[above right=2pt] {Obs 1: (2,2)};
        \filldraw[red] (3,1.5) circle (2pt) node[right=2pt] {Obs 2: (3, 1.5)};
    \end{tikzpicture}
    \end{center}
\pause $\sigma_{x1} = \sigma_{x2} = 1$ \\[1ex] Now which point?
\end{frame}

\begin{frame}
    \frametitle{}

    \begin{center}
    \begin{tikzpicture}[scale=1.5]
        % Draw grey grid lines
        \draw[gray, thin] (0,0) grid (3,3);
        
        % Draw axes
        \draw[->] (0,0) -- (3.5,0) node[right] {$x$};
        \draw[->] (0,0) -- (0,3.5) node[above] {$y$};
        
        % Add tick marks and labels for x-axis
        \foreach \x in {0,1,2,3}
            \draw (\x,0.05) -- (\x,-0.05) node[below] {\x_1};
        
        % Add tick marks and labels for y-axis
        \foreach \y in {0,1,2,3}
            \draw (0.05,\y) -- (-0.05,\y) node[left] {\x_2};
        
        % Add points
        \filldraw[black] (1,1) circle (2pt);
        \filldraw[blue] (2,2) circle (2pt) node[above right=2pt] {Obs 1: (2,2)};
        \filldraw[red] (3,1.5) circle (2pt) node[right=2pt] {Obs 2: (3, 1.5)};
    \end{tikzpicture}
    \end{center}
    \vspace*{-1cm}
\pause Suppose we add a new outlier data point to our dataset. We won't keep it when matching, but it means
\alert{$\sigma_{x1} = 3$} while \alert{$\sigma_{x2} = 1$} still.\\[1ex] 
Now which point? (Hint: $\sqrt{(\frac{1}{3}})^2 + 1^2)} > \sqrt{(\frac{2}{3}})^2 + 0.5^2)}$
\end{frame}


\begin{frame}
    \frametitle{}

    \begin{center}
    \begin{tikzpicture}[scale=1.5]
        % Draw grey grid lines
        \draw[gray, thin] (0,0) grid (3,3);
        
        % Draw axes
        \draw[->] (0,0) -- (3.5,0) node[right] {$x$};
        \draw[->] (0,0) -- (0,3.5) node[above] {$y$};
        
        % Add tick marks and labels for x-axis
        \foreach \x in {0,1,2,3}
            \draw (\x,0.05) -- (\x,-0.05) node[below] {\x_1};
        
        % Add tick marks and labels for y-axis
        \foreach \y in {0,1,2,3}
            \draw (0.05,\y) -- (-0.05,\y) node[left] {\x_2};
        
        % Add points
        \filldraw[black] (1,1) circle (2pt);
        \filldraw[blue] (2,2) circle (2pt) node[above right=2pt] {Obs 1: (2,2)};
        \filldraw[red] (3,1.5) circle (2pt) node[right=2pt] {Obs 2: (3, 1.5)};
    \end{tikzpicture}
    \end{center}
\vspace*{-1cm}
    \pause \small Our goal is to match observations with same \alert{potential outcomes}. Does this new point impact how $x$ impacts observation's potential outcomes? \\[0.5ex]
\pause Then why did we let it shape the point to which we matched?
\end{frame}

\begin{frame}
    \frametitle{Matching and ``Similarity''}

Matching doesn't assume functional forms like regressions.\\[1ex]
\pause But you do still have to \alert{define your similarity measure!} \\[1ex]
\pause And that requires choosing \alert{relative importance weights of different features}, even if that isn't explicitly obvious.
\end{frame}

\end{document}