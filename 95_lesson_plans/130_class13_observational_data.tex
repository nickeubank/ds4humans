% !TEX program = lualatex

\documentclass[12pt]{beamer}
% For handout add ,handout after 11pt

\usetheme{auriga}
\usecolortheme{auriga}

\usepackage{booktabs}
\usepackage{graphicx}
\usepackage[export]{adjustbox}
\usepackage{color}

% Set up emoji support with a fallback font
\newfontfamily\emojifont{Apple Color Emoji}[Renderer=Harfbuzz]
\DeclareTextFontCommand{\emoji}{\emojifont}

% Beamer list spacing
% \setlength{\leftmargini}{1.5em}
\setlength{\topsep}{1em}
\setlength{\itemsep}{1em}
\setlength{\parskip}{1.5em}
\setlength{\parsep}{0.5em}

% define some colors for a consistent theme across slides
\definecolor{red}{RGB}{181, 23, 0}
\definecolor{blue}{RGB}{0, 118, 186}
\definecolor{gray}{RGB}{146, 146, 146}

\title{Observational Data \& Regressions}

\author{Nick Eubank}

\date{\vspace*{.3in} \date}

\begin{document}
\setbeamercolor{page number in head/foot}{fg=white}

\begin{frame}[c]{}
    \titlepage
\end{frame}

\begin{frame}
    \frametitle{Observational Data}
No baseline differences: $E(Y^0|D=1) = E(Y^0|D=0)$ \\[1ex]
\begin{itemize}
    \pause \item Experiments: ensure with Law of Large Numbers
\end{itemize}
\pause If we can't randomize, then we don't get to rely on LLN. So then what? \\[1ex]
\pause Welcome to observational data!
\end{frame}

\begin{frame}
    \frametitle{Observational Data}
Suppose there \emph{are} baseline differences. \\[1ex]
\pause Suppose also that we understand the source of those differences! \\[1ex]
\pause Can we \emph{adjust} for those differences? \\[1ex]
\pause $E(Y^0|D=1, X) = E(Y^0|D=0, X)$ \\[1ex]
That's what causal inference with observational data is all about.
\end{frame}

\begin{frame}
    \frametitle{Omitted Variable Bias}
We want to know the effect of $D_i$ on $Y$. Suppose the causal relationship is:
    \begin{eqnarray}
    Y_i = \alpha^L + \beta^L D_i + \gamma A_i + e_i^L
    \end{eqnarray}
\pause But suppose we \emph{estimate}:
\begin{eqnarray}
Y_i =\alpha^S + \beta^S D_i + e_i^S
\end{eqnarray}
\pause Omitted Variable Bias is the difference between the true coefficient $\beta^L$ and the coefficient we estimate $\beta^S$.
\end{frame}

\begin{frame}
    \frametitle{Omitted Variable Bias}
    \begin{eqnarray}
    Y_i &=& \alpha^L + \beta^L D_i + \gamma A_i + e_i^L  \nonumber\\
Y_i &=&\alpha^S + \beta^S D_i + e_i^S \nonumber
\end{eqnarray}
    So what is $\beta^L - \beta^S$?
\pause 
    \begin{eqnarray}
\beta^L - \beta^S &=& \text{relation between }A_i\text{ and }D_i * \text{ Effect of } A_i \nonumber \\
\pause \beta^L - \beta^S &=& \pi_1 * \gamma \nonumber
    \end{eqnarray}
Where $\pi_1$ comes from estimating $A_i = \pi_0 + \pi_1D_i + \mu_i$
\end{frame}

\begin{frame}
    \frametitle{Omitted Variable Bias}
    \begin{eqnarray}
\beta^L - \beta^S &=& \text{relation between }A_i\text{ and }D_i * \text{ Effect of } A_i \nonumber \\
\beta^L - \beta^S &=& \pi_1 * \gamma \nonumber
\end{eqnarray}
Remember our 2x2 discussion of baseline differences $E(Y^0|D=1) = E(Y^0|D=0)$? \\[1ex]
\pause Asked for reason people with different $Y^0$ would sort differentially into $D=0$ and $D=1$. \\[1ex]
\pause In other words, we needed an unobserved factor ($A_i$) that was correlated with $D$ ($\pi_1 \neq 0$) and $Y^0$ ($\gamma \neq 0$).
\end{frame}

\end{document}