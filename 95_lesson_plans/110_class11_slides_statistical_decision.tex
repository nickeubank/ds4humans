% !TEX program = lualatex

\documentclass[12pt]{beamer}
% For handout add ,handout after 11pt

\usetheme{auriga}
\usecolortheme{auriga}

\usepackage{booktabs}
\usepackage{graphicx}
\usepackage[export]{adjustbox}
\usepackage{color}

% Set up emoji support with a fallback font
\newfontfamily\emojifont{Apple Color Emoji}[Renderer=Harfbuzz]
\DeclareTextFontCommand{\emoji}{\emojifont}

% Beamer list spacing
% \setlength{\leftmargini}{1.5em}
\setlength{\topsep}{1em}
\setlength{\itemsep}{1em}
\setlength{\parskip}{1.5em}
\setlength{\parsep}{0.5em}

% define some colors for a consistent theme across slides
\definecolor{red}{RGB}{181, 23, 0}
\definecolor{blue}{RGB}{0, 118, 186}
\definecolor{gray}{RGB}{146, 146, 146}

\title{Statistical Decision-Making \& Uncertainty}

\author{Nick Eubank}

\date{\vspace*{.3in} \date}

\begin{document}
\setbeamercolor{page number in head/foot}{fg=white}

\begin{frame}[c]{}
    \titlepage
\end{frame}


\begin{frame}[c]{Frequentist Hypothesis Testing}
\pause \alert{Binary} Decision Making: Reject Null Hypothesis if $p <$ threshold.
\begin{enumerate}
    \pause \item ``No effect'' is a strawman
    \pause \item Binary. But ``surely, God loves the .06 nearly as much as the .05.''
    \pause \item Conflated: depends on sample size \alert{and} effect size.
\end{enumerate}
\end{frame}



\begin{frame}[c]{Frequentist Hypothesis Testing}
``Statistical Significance'' does not mean \alert{substantively} significant. \\[1ex]
\begin{itemize}
    \pause \item $H_0$ of zero effect is a straw man.
    \pause \item Consequently, there will \alert{always exist} a sample size that will allow you to reject the null hypothesis of zero. Always.
\end{itemize}
\pause So clearly we can't just rely on hypothesis testing.
\end{frame}

\begin{frame}[c]{Evaluating Evidence}
Empirical research generates \alert{evidence}. \\[1ex]
\pause Interpretation of evidence requires \alert{judgment.}  \\[2ex]
\pause So how can we evaluate results more holistically?
\end{frame}

\begin{frame}[c]{Frameworks}
Frequentist:
\begin{itemize}
    \pause \item There is a single value of $\theta$\\
    \pause i.e., $\theta$ is not a random variable!\\
    \pause In binary setting: null hypothesis is true or not
    \pause \item But our estimates are uncertain because of sampling.
\end{itemize}
\pause Bayesian:
\begin{itemize}
    \pause \item $\theta$ is a random variable.
    \pause \item We are trying to estimate the distribution of likely values of $\theta$.
    \pause \item Explicitly accounts for prior beliefs.
\end{itemize}
\end{frame}

\begin{frame}[c]{Frameworks}
\begin{enumerate}
    \item \alert{Hypothesis Testing}
    \begin{itemize}
        \pause \item Frequentist.
        \pause \item Binary decision.
        \pause \item \emoji{🤮}
    \end{itemize}
    \pause \item \alert{Maximum Likelihood Estimation}
    \begin{itemize}
        \pause \item Still frequentist. \\
        \pause \item Gives estimate of ``most likely'' value.\\
        \pause \item Asymptotically, gives sampling distribution of estimate \alert{$Pr(\text{data}| \theta)$}.
    \end{itemize}
    \pause \item \alert{Bayesian}
    \begin{itemize}
        \item Matches intuitive idea: estimating \alert{$Pr(\theta|\text{data})$}.
    \end{itemize}
\end{enumerate}
\end{frame}

\begin{frame}[c]{Confidence Intervals}
\pause Suppose we move from \emph{Hypothesis Testing} framework to \emph{Maximum Likelihood} framework.\\[1ex]
\pause Under some assumptions, coefficient is MLE estimate $\theta$\\[1ex]
\pause Asymptotically, our estimate has a Normal sampling distribution:
$$\hat{\theta} \sim N(\theta, \sigma^2)$$
\pause Matching confidence intervals given in regression.\\[1ex]
So can we say that there's a 95\% probability that the true value of $\theta$ is inside that confidence interval? \\[1ex]
\pause \emph{Technically} no — it is, or it isn't.\\[1ex] 
Before you collect data, you can say there's a 95\% chance you'll draw data that generates an estimate with a confidence interval that includes true $\theta$. \pause \alert{But...}
\end{frame}

\begin{frame}[c]{Bernstein-von Mises Theorem}
The distribution of the Maximum Likelihood Estimator ($Pr(data|\theta)$) converges to the Bayesian posterior distribution $Pr(\theta|data)$ as $n\rightarrow \infty$ given ``reasonable'' priors.\\[1ex]

\pause So in small sample you have to be careful, but treating confidence intervals \alert{as if} they are the distribution of the likely true value of a coefficient is \alert{defensible}.
\end{frame}

\begin{frame}[c]{Bernstein-von Mises Theorem}
How fast does MLE Sampling Distribution converge to Bayesian Posterior Distribution?\\[2ex]
Depends, but in general faster when priors are weak (meaning before you ran your analysis, you would put similar likelihood on all possible values of $\theta$).
\begin{itemize}
    \pause \item The stronger your priors, the slower the convergence.
\end{itemize}
\end{frame}

\begin{frame}[c]{Bernstein-von Mises Theorem}
Why? \\[2ex]
\pause \alert{Priors is information that sets Frequentist and Bayesian stats apart}. \\[1ex]
\pause But eventually, enough data will always overwhelm your priors. \\[1ex]
\pause At that point, Frequentist and Bayesian begin to converge.
\end{frame}


\begin{frame}[c]{Summary}
\begin{enumerate}
    \item Hypothesis Testing
    \begin{itemize}
        \pause \item Defined in small samples, but very problematic.
        \pause \item Binary, conflates magnitude and precision.
    \end{itemize}
    \pause \item Maximum Likelihood
    \begin{itemize}
        \pause \item Still frequentist, still kinda incoherent.
        \pause \item \emph{But} it converges to Bayesian as $n\rightarrow \infty$
    \end{itemize}
    \pause\item Bayesian
    \begin{itemize}
        \pause \item What you're thinking of intuitively.
    \end{itemize}
\end{enumerate}
\end{frame}

\begin{frame}[c]{Summary}
    \alert{In large samples}, treating confidence intervals \alert{as if} they are the distribution of the likely true value of a coefficient is \alert{defensible}.\\[1ex]
    \pause Caveats:
    \begin{itemize}
        \pause \item Assumes MLE properly specified (ignores model uncertainty).
        \pause \item Not true in small samples!
    \end{itemize}
\end{frame}
\end{document}