% !TEX program = lualatex

\documentclass[12pt]{beamer}
% For handout add ,handout after 11pt

\usetheme{auriga}
\usecolortheme{auriga}

\usepackage{booktabs}
\usepackage{graphicx}
\usepackage[export]{adjustbox}
\usepackage{color}

% Set up emoji support with a fallback font
\newfontfamily\emojifont{Apple Color Emoji}[Renderer=Harfbuzz]
\DeclareTextFontCommand{\emoji}{\emojifont}

% Beamer list spacing
% \setlength{\leftmargini}{1.5em}
\setlength{\topsep}{1em}
\setlength{\itemsep}{1em}
\setlength{\parskip}{1.5em}
\setlength{\parsep}{0.5em}

% define some colors for a consistent theme across slides
\definecolor{red}{RGB}{181, 23, 0}
\definecolor{blue}{RGB}{0, 118, 186}
\definecolor{gray}{RGB}{146, 146, 146}

\title{Experiments and External Validity}

\author{Nick Eubank}

\date{\vspace*{.3in} \date}

\begin{document}
\setbeamercolor{page number in head/foot}{fg=white}

\begin{frame}[c]{}
    \titlepage
\end{frame}

\begin{frame}[c]{External Validity}
External validity is the extent to which \alert{the effect measured in a study} is a good estimate for \alert{the effect in a different context.} \\[1ex]
\pause Do the results \emph{generalize.} \\[2ex]
\pause External validity can depend on many things:
\begin{itemize}
    \pause \item Are the entities in the study drawn from the same population in the new setting?
    \pause \item Is the treatment the same in the new setting?
    \pause \item Is the broader context the same in the new setting (artificiality)?
\end{itemize}
\end{frame}

\begin{frame}[c]{External Validity: Sample Composition}
White men \alert{over-represented} in medical studies.
\begin{itemize}
    \pause \item Cardiac drug doses often too high for women.
    \pause \item Other cardiac drugs interact poorly with gene common in Asian and Pacific Islanders.
    \pause \item Turns out Multiple Sclerosis driven by a different mutation in Black than European descendants.
\end{itemize}
\pause Psychology studies mostly run on Western, Educated students from Industrialized, Rich, and Democratic countries (they're WEIRD). \\[1ex]
\end{frame}

\begin{frame}[c]{External Validity: Treatment Differences}
A new car dealer sends out mailers offering \$3,000 off list.\\[1ex]
Customers who receive more likely to come buy car in following 6 months. \\[1ex]
\pause Will that generalize to:
\begin{itemize}
    \item \$1,000 discount?
    \item Mailers from used car dealership?
    \item Coupons handed out by hand?
\end{itemize}
\end{frame}


\begin{frame}[c]{External Validity: Artificiality}
Often a \alert{tension between internal and external validity.} \\[1ex]
\pause More control (in lab, measure everything, etc.), the more likely to get internal validity. \\[1ex]
\alert{BUT:} \\[1ex]
May create an artificiality that reduces external validity.\\[1ex]
\pause (One explanation for medical study issues)
\begin{itemize}
    \item \alert{Hawthorne Effect}: People behave differently when know observed. \\[1ex]
    \pause Originally documented in studies of workers.  
\end{itemize}
\end{frame}

\begin{frame}[c]{External Validity}
But enough of the easy causes...
\end{frame}


\begin{frame}[c]{External Validity: Exercise 1}
Durham Hard Cider Company (DHCC) has a new hard cider. Available Durham restaurants, but sales are weak. \\[1ex]

 \pause DHCC offered free samples in randomly selected Durham restaurants. Sales of new cider in those restaurants \alert{jumped 30\% in following weeks}. \\[1ex]

\begin{itemize}
    \pause \item Do you think free samples would have \alert{the same effect in Chapel Hill?} \\[1ex] 
    In other words, do you think this result is \alert{likely to generalize/have high external validity to Chapel Hill?}
    \pause \item Do you think free samples would have \alert{the same effect for Bud Light?}
\end{itemize}
\end{frame}

\begin{frame}[c]{External Validity}
The external validity of a study is not \alert{singular}. \\[1ex] 
\pause External validity is always defined \alert{with respect to the context to which one wishes to generalize the results.} \\[2ex] 

\pause The same study may have very \alert{high} external validity to the context of one business, but \alert{low} external validity to the context of another business. \\[1ex]
\pause Distinguishes internal validity from external validity.
\end{frame}


\begin{frame}[c]{External Validity: Exercise 2a}
Apple offers a \$300 discount on laptops in a randomly selected set of cities for 24 hours. \\[1ex]

Sales on day of the sale so good (compared to sales in other cities), Apple considering lowering list prices. 

\begin{itemize}
    \pause \item Do you think sales would increase the same amount \alert{if Apple reduced prices everywhere?}
\end{itemize}
\end{frame}


\begin{frame}[c]{External Validity: Exercise 2b}
A national used car dealership chain offers \$1,000 discount at locations in a random set of cities for 24 hours. \\[1ex]

Sales on day of the sale so good (compared to sales in their other locations), company considering lowering list prices. 

\begin{itemize}
    \pause \item Do you think sales would increase the same amount \alert{if the dealer reduced prices everywhere?}
\end{itemize}
\end{frame}

\begin{frame}[c]{External Validity: Time-shifting \& Competition}
Primacy and novelty effects are the only reason short-term studies may not generalize. \\[1ex]
\pause Sales and promotions have multiple effects on consumers.
\begin{itemize}
    \pause \item Win customers from competitors \\[0.5ex]
    \pause Real gains (though competitors may eventually respond)
    \pause \item Time-shift purchases \\[0.5ex]
    \pause Illusory gains.
\end{itemize}
\pause Relative magnitudes depend on produce and market structure.
\end{frame}

\begin{frame}[c]{External Validity: Exercise 3a}
Bing has launched a new algorithm that makes search results slightly worse. \\[1ex]

A/B testing shows that because users don't get the ``right'' result as quickly, they tend to search more and thus see more ads, increasing revenue!

\begin{itemize}
    \pause \item If you ran Bing, would you adopt this new algorithm?
\end{itemize}
\end{frame}


\begin{frame}[c]{External Validity: Exercise 3b}
Google has launched a new algorithm that makes search results slightly worse. \\[1ex]

A/B testing shows that because users don't get the ``right'' result as quickly, they tend to search more and thus see more ads, increasing revenue!

\begin{itemize}
    \pause \item If you ran Google, would you adopt this new algorithm? \\[1ex]
    \pause \item Guess what? They did! (According to a lot of docs and reporting from google trial)
\end{itemize}
\end{frame}

\begin{frame}[c]{External Validity: Exercise 4}
Google is revising its algorithm (to actually help customers). \\[1ex]

Google has found good recipe sites (non-spam sites) tend to have additional, novel prose around the recipe. \\[1ex]

A/B testing shows promoting these pages increases user searches for recipes (including searches for \emph{new} recipes) over 1 month period. \\[1ex]

\begin{itemize}
    \pause \item Do you think this result is likely to hold if rolled out globally \alert{over the long run?} Why or why not?
\end{itemize}
\end{frame}

\begin{frame}[c]{External Validity: General Equilibrium Effects}
In reading, discussed Rajasthani nurses.\\[1ex]
\pause \alert{Adversarial users} are people who \alert{change their behavior in response to the deployment of an algorithm} specifically to take advantage of the system. \\[1ex]
\pause Search Engine Optimization (SEO) is one of the quintessential examples. \\[2ex]
We'll talk more about this later this semester — it comes up a \emph{lot} in machine learning — but other examples include:

\begin{itemize}
    \item Financial fraud: keeping purchases under \$10,000, or stealing tiny amounts from lots of accounts.
    \item Content policing: ``unalive,'' ``seggs''
    \item Surge pricing: drivers who coordinate to create apparent driver ``shortages'' by all signing off.
\end{itemize}
\end{frame}

\begin{frame}[c]{External Validity: Recap}
    \begin{enumerate}
    \pause\item External validity is not singular.
    \pause \item Simple considerations: 
        \begin{itemize}
            \pause \item Sample composition
            \pause \item Treatment substance
            \pause \item Broader context/artificiality
        \end{itemize}
    \pause \item Short-versus-long term
        \begin{itemize}
            \pause \item Primacy and novelty effects
            \pause \item Time-switching v. competition
        \end{itemize}
    \pause \item General Equilibrium Effects
        \begin{itemize}
            \item Political Responses (Rajasthan)
            \item Adversarial Users
        \end{itemize}
    \pause \item Others?
    \end{enumerate}
\end{frame}

\end{document}