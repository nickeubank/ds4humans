% !TEX program = lualatex

\documentclass[12pt]{beamer}
% For handout add ,handout after 11pt

\usetheme{auriga}
\usecolortheme{auriga}

\usepackage{booktabs}
\usepackage{graphicx}
\usepackage[export]{adjustbox}
\usepackage{color}

% Beamer list spacing
% \setlength{\leftmargini}{1.5em}
\setlength{\topsep}{2em}
\setlength{\itemsep}{0pt}
\setlength{\parskip}{0pt}
\setlength{\parsep}{0pt}


% define some colors for a consistent theme across slides
\definecolor{red}{RGB}{181, 23, 0}
\definecolor{blue}{RGB}{0, 118, 186}
\definecolor{gray}{RGB}{146, 146, 146}


\title{Welcome to IDS 701!}

\author{Nick Eubank}

\begin{document}

\setbeamercolor{page number in head/foot}{fg=white}

\begin{frame}[c]{}

{\LARGE Welcome to IDS 701!}

\end{frame}

\begin{frame}[c]{Exercise 1}

You are all enrolled in a Data Science Master's Degree. But what \alert{is} Data Science? \\

\vspace*{0.5cm}
\pause In your group, please:

\begin{enumerate}
    \item Define ``data science''
    \item Tell us what you hope to be able to do when you graduate (that you couldn't do before)?
\end{enumerate}

\end{frame}


\begin{frame}[c]{What is data science (in this class)?}

\pause    Study of how to \alert{solve problems} \pause by answering questions \pause using quantitative methods.

\end{frame}

\begin{frame}[c]{Solve problems ``by answering questions''?}

\pause All data science tools can be viewed as methods of answering questions about the world (as embodied in your data).\\

\vspace*{0.5cm}

\pause True in both a narrow technical sense:

$$\operatorname*{argmin}_{a \in \text{possible answers}} \text{Loss Fn}(a)$$

\pause Also true in a more subtle but important sense we'll discuss more later.

\end{frame}


\begin{frame}[c]{Solving Problems with Data Science}
\begin{enumerate}
    \pause \item Work with stakeholder to articulate the problem you wish to solve.
    \pause \item Determine what question, if answered, would help address your stakeholder's problem.
    \begin{itemize}
        \pause \item Descriptive Questions: Overall distribution of features in the world. \\
        $\rightarrow$ Useful for prioritizing efforts.
        \pause \item (Passive) Prediction Questions: Future or unknown outcomes for individuals \\
        $\rightarrow$ Useful for identifying individuals for additional attention (patients likely to experience complications) \\
        $\rightarrow$ Useful for automation (how would a human rate this job applicant?)
        \pause \item Causal Questions: \alert{The first half of this class!}
    \end{itemize}
\end{enumerate}
\end{frame}

\begin{frame}[c]{Exercise 2: Causal Questions}
\pause In your group, please:

\begin{enumerate}
    \item Tell us what types of problems we can solve by answering causal questions?
    \item Why is prediction (e.g., supervised machine learning or predicted values from regressions) not enough?
\end{enumerate}
\end{frame}

\begin{frame}[c]{Causal Questions}
Answering causal questions entails predicting the likely consequence of \alert{taking an action}. \\

\vspace*{0.5cm}

\pause When we take an action, we are changing the \alert{data generating process} (the underlying process that generates whatever data we may be feeding into our model). This disruption means that correlations that were present in the data previously may no longer apply.
\end{frame}


\begin{frame}[c]{Why is Prediction Not Enough?}
Suppose you run a hotel chain. You want to know the likely effect of \alert{raising room rates on occupancy rates.}\\
\vspace*{0.5cm}
\pause You collect data on hotel prices and occupancy rates for the past few years. \\
\vspace*{0.5cm}
\pause What do you think you'd find?
\end{frame}


\begin{frame}[c]{Why is Prediction Not Enough?}
Higher prices would be correlated with higher occupancy rates!\\
\vspace*{0.5cm}
\pause Do you think raising \emph{your} prices would increase \emph{your} occupancy rates?\\
\vspace*{0.5cm}
\pause Why is this past pattern not holding? \\
\vspace*{0.5cm}
In the historic data, prices are high because something external was driving up demand. But that is no longer true if you raise your rates unilaterally.
\end{frame}

\begin{frame}[c]{Why is Prediction Not Enough?}

\begin{quote}
Correlation does not imply causation.    
\end{quote}

\end{frame}


\begin{frame}[c]{Why is Prediction Not Enough?}

\begin{quote}
Correlation does not \alert{necessarily} imply causation.    
\end{quote}

\end{frame}

\begin{frame}[c]{Why is Prediction Not Enough?}
\begin{quote}
Correlation does not \alert{necessarily} imply causation.    
\end{quote}
Causal inference is about understanding \alert{when} correlation \alert{does} imply causation, and why.
\end{frame}

\begin{frame}[c]{Class Structure}
\begin{enumerate}
    \item Causal Questions:
    \begin{itemize}
        \item Theory of Causal Inference
        \item Experiments
        \item Non-Experimental (Observational)
    \end{itemize}
    \item Project Design (Solving Problems):
    \begin{itemize}
        \item Working with Stakeholders
        \item Types of Questions
    \end{itemize}
\end{enumerate}
\end{frame}


\begin{frame}[c]{Class Structure}
\begin{enumerate}
    \item Largely Flipped:
    \begin{itemize}
        \item Read Before Class
        \item Quizzes
    \end{itemize}
    \item Website: ds4humans.com
\end{enumerate}
\end{frame}

\begin{frame}[c]{Laptop Policy}
Laptops (phones, etc.) will \alert{not} be allowed in most classes. \\

\vspace*{0.5cm}

\pause We have very clear, experimental evidence that:
\begin{itemize}
    \item Even just the \emph{temptation} to jump to social media, etc. is distracting.
    \item Laptops don't just hurt your learning — they harm the learning of those around you.
    \item Taking notes by hand is harder and slower... and that makes it work better.
\end{itemize}
\end{frame}



\end{document}