% !TEX program = lualatex

\documentclass[12pt]{beamer}
% For handout add ,handout after 11pt

\usetheme{auriga}
\usecolortheme{auriga}

\usepackage{booktabs}
\usepackage{graphicx}
\usepackage[export]{adjustbox}
\usepackage{color}
\usepackage{tikz}

% Set up emoji support with a fallback font
\newfontfamily\emojifont{Apple Color Emoji}[Renderer=Harfbuzz]
\DeclareTextFontCommand{\emoji}{\emojifont}

% Beamer list spacing
% \setlength{\leftmargini}{1.5em}
\setlength{\topsep}{1em}
\setlength{\itemsep}{1em}
\setlength{\parskip}{1.5em}
\setlength{\parsep}{0.5em}

% define some colors for a consistent theme across slides
\definecolor{red}{RGB}{181, 23, 0}
\definecolor{blue}{RGB}{0, 118, 186}
\definecolor{gray}{RGB}{146, 146, 146}
\newcommand{\indep}{\perp\!\!\!\perp}
\title{Curse of Dimensionality, \\ Propensity Scores \& Matching}

\author{Nick Eubank}

\date{\vspace*{.3in} \date}

\begin{document}
\setbeamercolor{page number in head/foot}{fg=white}

\begin{frame}[c]{}
    \titlepage
\end{frame}

\begin{frame}
    \frametitle{Curse of Dimensionality}
Assume we are doing Coarse Exact Matching on 200 observation dataset. \alert{100 obs Treated, 100 obs control}.  \\[1ex]

Assume two covariates, $x_1 \in {1, 2, 3, 4}$ and $x_2 \in {1, 2, 3, 4}$. 

\pause Assume uniformly distributed. How many T and how many C in each matching ``bucket''?

\pause $\frac{100}{4^2} = \frac{100}{16} = 6.25$ T and $6.25$ C.\\[1ex]
\pause We're matching, so probably \emph{not} uniform, but prob fine.
\end{frame}

\begin{frame}
    \frametitle{Curse of Dimensionality}
Assume we are doing Coarse Exact Matching on 200 observation dataset. \alert{100 obs Treated, 100 obs control}.  \\[1ex]

Assume \alert{three} covariates, $x_1 \in {1, 2, 3, 4}$, $x_2 \in {1, 2, 3, 4}$, $x_3 \in {1, 2, 3, 4}$. 

Assume uniformly distributed. How many T and how many C in each matching ``bucket''?

\pause $\frac{100}{4^3} = \frac{100}{64} = 1.5$ T and $1.5$ C.\\[1ex]
\pause Sure to be plenty of empty cells, but also cells that match.
\end{frame}

\begin{frame}
    \frametitle{Curse of Dimensionality}
Assume we are doing Coarse Exact Matching on 200 observation dataset. \alert{100 obs Treated, 100 obs control}.  \\[1ex]

Assume \alert{four} covariates, $x_1 \in {1, 2, 3, 4}$, $x_2 \in {1, 2, 3, 4}$, $x_3 \in {1, 2, 3, 4}$ and $x_4 \in {1, 2, 3, 4}$.

Assume uniformly distributed. How many T and how many C in each matching ``bucket''?

\pause $\frac{100}{4^4} = \frac{100}{256} = 0.4$ T and $0.4$ C.\\[1ex]
\pause Oof... mostly empty cells?
\end{frame}


\begin{frame}
    \frametitle{Curse of Dimensionality}
Assume we are doing Coarse Exact Matching on 200 observation dataset. \alert{100 obs Treated, 100 obs control}.  \\[1ex]

Assume \alert{five} covariates, $x_1 \in {1, 2, 3, 4}$, $x_2 \in {1, 2, 3, 4}$, $x_3 \in {1, 2, 3, 4}$ and $x_4 \in {1, 2, 3, 4}$, $x_5 \in {1, 2, 3, 4}$.

Assume uniformly distributed. How many T and how many C in each matching ``bucket''?

\pause $\frac{100}{4^5} = \frac{100}{1,025} = 0.1$ T and $0.1$ C.\\[1ex]
\pause Oof... Even if most are in an area of overlap, getting \emph{real} hard.
\end{frame}


\begin{frame}
    \frametitle{Curse of Dimensionality}
Even with just \alert{five} covariates, each with only \alert{four discrete values (no continuous variables!)}, finding matches gets quickly fast! \\[1ex]
\pause \alert{Curse of Dimensionality}: adding dimensions increases the distance between points fast. 
\pause Challenge with matching on covariates \\[1ex]
(in theory and, historically, in practice. Can be hard computationally!) \\
\vspace*{1cm}
\pause Thus the allure of \alert{propensity score matching}.
\end{frame}



\begin{frame}
    \frametitle{Conditional Independence Assumption}
As discussed before, our goal in observational causal inference is to find and condition on covariates $X$ such that:
\begin{equation*}
    E(Y^{0} | D_i=1, X) = E(Y^{0} | D_i=0, X)
\end{equation*}
Statisticians generally use the slightly stronger \alert{Conditional Independence Assumption} (which implies No Baseline Differences): \\[1ex]
\begin{equation*}
(Y^1, Y^0) \indep D | X
\end{equation*}
So when you see $(Y^1, Y^0) \indep D | X$, that implies no baseline differences (and also no differential treatment effects).
\end{frame}


\begin{frame}
    \frametitle{Propensity Scores}
\alert{Propensity Score Theorem}: 
\begin{eqnarray*}
    \text{if } (Y^1_i, Y^0_i) \indep D_i | X \\
    \text{then} (Y^1_i, Y^0_i) \indep D_i | p(X_i) \\
\end{eqnarray*}
Where $p(X_i)$ is the propensity score: $p(X) \equiv Pr(D_i=1 | X_i)$ \\[1ex]
\pause Mathematical Proof in Angrist and Pischke, \emph{Mostly Harmless Econometrics}, Sec 3.3.2. or \url{https://youtu.be/nnC1tlircdE}
\end{frame}

\begin{frame}
    \frametitle{Propensity Scores Matching}
Why does conditioning on Propensity Score $\rightarrow$ No Baseline Differences? \\[1ex]
\pause  Matching on covariates $X_i$ makes intuitive sense: \\[1ex]
\indent Pair T and C observations that look the same and thus (we hope) have as close to the same \alert{potential outcomes} ($Y^0_i, Y^1_i$) as possible. \\[2ex]
\end{frame}

\begin{frame}
    \frametitle{Propensity Scores Matching}
Why does conditioning on Propensity Score $\rightarrow$ No Baseline Differences? \\[1ex]
\pause What does matching on $Pr(D_i = 1|X_i)$ accomplish? \\[1ex]
\pause There's no $Y^0_i$ in that equation! \\[1ex]
Baseline differences are violated when a factor ($X_i$) is correlated with $Y^0_i$ and $D_i$. \\[1ex]
\pause Propensity Score Matching \alert{just focuses on breaking correlation between factors $X_i$ and $D_i$.} 
\end{frame}

\begin{frame}
    \frametitle{Propensity Scores Matching}
Strengths:
\begin{enumerate}
    \pause \item Matching on a scalar is easy!
    \pause \item Asymptotically, it breaks the correlation between $X_i$ and $D_i$.
\end{enumerate}
\pause Weakness: 
\begin{enumerate}
    \item You \emph{aren't} trying to make pairs as comparable as possible in terms of $Y^0_i, Y^1_i$. \\
    \begin{itemize}
        \pause \item Thus sample left after matching may be unbalanced and inefficient.
    \end{itemize} 
    \pause \item Proofs assume you \emph{know} $Pr(D_i=1|X_i)$. Issues when estimating.
\end{enumerate} 
\end{frame}

\begin{frame}
    \frametitle{My (Opinionated) Summary}
\begin{itemize}
    \item If you have data to match on covariates, do.
    \item Propensity score matching is an OK choice if (a) you have a strong understanding of the true $Pr(D_i=1|X_i)$ and (b) you aren't dropping too many observations (potentially giving rise to imbalance).
\end{itemize}
\end{frame}

\end{document}